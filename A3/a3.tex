\documentclass{article}
\usepackage{amsmath}
\usepackage{amssymb}
\usepackage{algorithm}
\usepackage{float}
\usepackage{color}
\usepackage{multicol}
\usepackage{forloop}
\usepackage{graphicx}
\usepackage[margin=0.8in]{geometry}
\usepackage{caption}
\usepackage{enumerate}

\graphicspath{ {.} }
\title{COMP 4106\\
	\large{Assignment 3, Reinforcement Learning}}
\author{Krystian Wojcicki, 101001444}
\date{Winter 2020}

\begin{document}
\maketitle

\section{Introduction}
Four different schemes were used. Tsetlin, Krinsky, Krylov and $L_{RI}$. I assumed the noise parameter to have a sigma of 1.0 and for the $Q$ function to map $Q(i) = i$. For each the hyper parameters were attempted to keep similar in order to do a fair comparison between the 4.

\section{Tsetlin}
 For the Tsetlin automata a memory value of 5 was utilized. With a transient time for learning of 10,000 and then 1000 iterations in order to test the accuracy. The accuracy was as follows $[0.9818, 0.0062, 0.0032, 0.0029, 0.0029, 0.003 ]$ with a speed of converge equal to $44.25$ across an ensemble average of 100 experiments. 

\section{Krinsky}
 For the Krinsky automata a memory value of 5 was utilized. With a transient time for learning of 10,000 and then 1000 iterations in order to test the accuracy. The accuracy was as follows $[9.877e-01, 8.100e-03, 9.000e-04, 1.000e-03, 1.200e-03, 1.100e-03]$ with a speed of converge equal to $43.25$ across an ensemble average of 100 experiments. 

\section{Krylov}
 For the Krinsky automata a memory value of 5 was utilized. With a transient time for learning of 10,000 and then 1000 iterations in order to test the accuracy. The accuracy was as follows $[0.9584, 0.04,   0.,     0.,     0.,     0.0016]$ with a speed of converge equal to $112.75$ across an ensemble average of 100 experiments. 

\section{$L_{RI}$}
 For the $L_{RI}$ automata a learning rate of $0.03\%$ was utilized. With a transient time for learning of 10,000 and then 1000 iterations in order to test the accuracy. The accuracy was as follows $[9.98e-01, 1.80e-03, 1.00e-04, 0.00e+00, 0.00e+00, 1.00e-04]$ with a speed of converge equal to $234.5$ across an ensemble average of 100 experiments. 

\section{Conclusion}
Overall all the models had very similar accuracy with $L_{RI}$ having a slight edge with Krylov relatively behind the rest of the bunch. In terms of speed there was a large difference, $L_{RI}$ took much longer to converge compared to the other models. Krylov also took a relativly large time to converge, but significantly quicker than $L_{RI}$. Tsetlin and Krinsky had very similar convergence. 

\end{document}